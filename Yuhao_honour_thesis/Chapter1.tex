\chapter{Introduction}
\label{Chapter1}
\section{Background and Motivation}

%%%%%%%%%%%%%%%% Next Version
%%%%%%%%%%%%%%%%

%\textbf{Why Bayesian Lasso problem}
%Advantages: 1. Incorporate Variation (inferential quantity), bring prior if we haev strong prior , %2. Bayesian: Potential for auotomatic tuning selection while Freq cross validation)
%Disadvantages:
%Lasso -> bayesian lasso -> MCMC approximate bayesian inference

\textbf{Introduction of Lasso Problem}\\
The Least Absolute Selection and Shrinkage Operator(Lasso) regression proposed by \cite{tibshirani_1996} belongs to one of the shrinkage methods. As one of the most traditional shrinkage methods, Lasso regression has been proven for his success in Statistical Community over the years.

The Lasso serves as two purposes, one is the estimation of regression parameter, the other is to effective shrinking of the coefficients to achieve variable selection purpose, which is also the fundamental difference of Lasso with other methods. The Lasso Regression is helpful particularly for high-dimensional data because of its sparsity nature.

\textbf{Explain purpose of Lasso}
The definition of linear regression model of interest can be referred based on the following definition defined by \cite{tibshirani_1996}: the $n \times 1$ vector of regression coefficient $\beta$,  $y$ is the response variable with a dimension of $n \time 1$, $X$ is data matrix after standardization with a dimension of $n \times p$ , $\mu$ is the population mean with a dimension of $n \times 1$, $\epsilon$ is independent and identically distributed normal noise with expectation of 0 and variance of $\sigma^2$. Then the linear model can be explained by the Equation \ref{eq:LRmodel}
\begin{equation}
	\label{eq:LRmodel}
	y = \mu 1_n + X\beta + \epsilon.
\end{equation} 
The Least square estimator suggest the sum of square of the difference between estimated response variable and true response variable should be used as loss function as described in Equation (\ref{eq:OLS})
\begin{equation}
	\label{eq:OLS}
	\hat{\beta} = \underset{\beta}{\operatorname{argmin}} || (\tilde{y} - X\beta)^T(\tilde{y}-X\beta)||.
\end{equation}
 The Lasso estimate of regression coefficient is based on Equation (\ref{eq:lasso1}), where the main distinction of Lasso is adding a penalty term of absolute value of regression coefficients $\beta$ in addition to the sum of squared value of residuals from ordinary regression objective function. The value of tuning parameter $\lambda$ is served as a measure of the extent of penalization.
 \begin{equation}
 	\label{eq:lasso1}
 	\hat{\beta}_{lasso} = \underset{\beta}{\operatorname{argmin}} (\tilde{y}-X\beta)^T(\tilde{y}-X\beta) + \lambda ||\beta||_1, \lambda \geq 0. \tilde{y} =  y - \bar{y}\textbf{1}_n \\
 \end{equation}
 Larger penalization leads to more sparse solution of regression coefficient due to a square constraint set results from $L_1$ penalty function, so that achieve variable selection of parameter, generating a higher prediction accuracy as well as interpretable model since we could drop the estimated regression coefficient that has 0 and state they have weak effect for prediction according to \cite{tibshirani_1996}.
However, due to non-existence of derivative of absolute value of regression coefficient $\beta$, alternative improved algorithm have been purposed and deployed such as Least Angle Regression(LARS), iterative soft-thresholding, subgradient method, and iteratively reweighted least square(IRLS) by \cite{efron_hastie_johnstone_tibshirani_2004},   \cite{beck_teboulle_2009}, \cite{nan_zhang_shuqing_zeng_2005} and \cite{friedman_hastie_tibshirani_2010}.

\textbf{Bayesian Lasso}
One of the detriments of the ordinary lasso is that the variation of inferential quantity can't be captured properly. To resolve this issue, \cite{tibshirani_1996} also suggests that the lasso estimate can also be extended under the Bayesian framework, which can be described as posterior mode Equation(\ref{eq:MLELasso}) if independent and identically distributed Laplacian prior from Equation (\ref{eq:LassoPrior}) is assigned,together with the likelihood form on Equation(\ref{eq:lassolikelihood}). 
\begin{equation}
	\label{eq:MLELasso}
	\hat{\beta}_{lasso} = \underset{\beta}{\operatorname{argmax}}P(\beta|y,\sigma^2,\tau), \tau = \frac{\lambda}{2\sigma^2}
\end{equation}
\begin{equation}
	\label{eq:LassoPrior}
	f(\beta_j) = (\frac{\tau}{2})^p exp(-\tau|\beta_j|),
\end{equation}
\begin{equation}
	\label{eq:lassolikelihood}
	p(y |\beta,\sigma^2) = N(y|X\beta,\sigma^2I_n).
\end{equation}
% Add full models
\textbf{Introduction of Bayesian Lasso Problem(Why Bayesian Lasso)}
Nevertheless, there is no tractable integration form for the Bayesian Lasso posterior until \cite{park_casella_2008} further explore the Lasso model under the setting of Bayesian framework, where the choice of a conditional Laplace prior distribution over the regression coefficient $\beta$ conditioning by standard error $\sigma^2$ is added to the Lasso penalty formulation in the frequentist framework to ensure unimodality of full posterior distribution. Based on closed form of tractable posterior distribution, a three-step Gibbs sampler is proposed to draw approximate samples from Bayes Lasso posterior distribution, which can be utilized for further inference of parameter of interest.
\begin{equation}
	\label{eq:lassoprior}
	\pi(\beta |\sigma^2) = \prod_{j=1}^p \frac{\lambda}{2\sqrt{\sigma^2}} e^{-\lambda|\beta_j|/\sqrt{\sigma^2}}
\end{equation}
There are several benefits for using the Bayesian Lasso model. Firstly, it has easier implementation than the traditional Lasso, although more computation of conditional distribution form is demanded. Secondly Bayesian credible interval of parameters can be generated simultaneously for modelling uncertainty and therefore can also guide variable selection based on the interpretation that lasso estimated is regarded as the mode of posterior distribution of $\beta$. 
Thirdly, \cite{park_casella_2008} also state that the Bayesian Lasso model could also be a potential solution for addressing the issue of attaining optimal tuning parameter $\lambda$ by marginal maximum likelihood method together with a sue of an suitable hyperprior such as gamma prior on the square of tuning parameter $\lambda$: $\lambda^2$. 
It could support a more stable automatic tuning process of choosing the most appropriate tuning parameter $\lambda$ for the Lasso problem, as opposed to inefficient $K$-fold cross validation approach under the ordinary Lasso model that is time consuming and computational demanding. Lastly, the three-step Bayesian Lasso Gibbs Sampler proposed by \cite{park_casella_2008} would yield an exact posterior distribution that can be sampled given exact form of conditional distribution of each model parameter conditioning on rest of the other parameters. Theoretically, \cite{khare_hobert_2013} has demonstrated a Bayesian Lasso Gibbs sampler version of central limit theorem(CLT), indicating Bayesian Lasso Gibbs Sampler satisfy geometriclly ergodicity if any values of sample size $n \geq 3$ and arbitary number of regression coefficient $p$, data matrix $X$, tuning parameter $\lambda$ are assigned. This means, the Bayesian lasso Gibbs Sampler is able to achieve asymptotically uncertainty of posterior estimation. 


\textbf{Challenges of Bayesian Lasso Problem}
On the other hand, the three-step Gibbs sampler from Markov Chain Monte Carlo(MCMC) under the class of stochastic approximation algorithm is time-consuming and computational challenging, given the fact that it normally requires a long time to converge inisde the interval of an acceptable tolerance, especially when $n$ is small and $p$ is large \cite{rajaratnam_sparks_2015}. 

% Collapsed Gibbs sampler / Blocked Gibbs Sampeler

% TBC


% A direct chaaracterization of Bayesian Lasso Gibbs Sampler is introduced
%Some improvement paper over the year(Variational Infernece)%

\textbf{Approximation Algorithm: Deterministic type}
Despite of the Bayesian Lasso Gibbs Sampler, alternative methods such as determinstic Approximate Bayesian Infernece methods has becoming popular due to its fast speed and simple computaation. There are two genres of Approximate Bayesian Inferenece method, which includes stochastic approach such as MCMC, where an exact result can be obtained if infinte computataional resource is assigned. Another category lies in deterministic approach that as an faster substitution compared with stochastic approximation approaches. Numerous algorithms have been designed and utilized widely such as Variational Bayes, Expectation Propagation algorithms etc. Deterministic approaches assume the approximation originates from a tractable distribution first and attempt to search for the distribution from this family that is the closest to the target posterior distribution by optimization techniques, it has been indicated that Variational Inference algorithm shows a descent computation cost and time-efficiency. 

\textbf{Variational Bayes}
The most traditional Variational Inference algorithm is known as Mean-Field Variational Bayes motivated by mean-field Theory in statistical physics yielded by 
\cite{jordan_ghahramani_jaakkola_saul_1998} and \cite{attias_1999}, which assume the approximated distribution is from independent product of parameter distribution.
Meanwhile, disadvantages of Variational Bayes include inexact approximation result under some scenarios. For example, it is suggested by \cite{bishop_2006} 
%Might have problem with citation pRML%
 that Variational Inference algorithm might underestimate the covariance between parameter of interest, if parameter of interest have a strong correlation. We will expand properties and derivation of Variational Inference more in subsection \ref{VI}.\\

\textbf{Drawbacks of VI}



\textbf{Motivation}
Motivated by the intention of further enhancing the approximation accuracy of Variational Bayes, we have designed two new Variational algorithms, particularly for Bayesian Lasso problem. By utilizing and fitting a Lasso distribution to marginal distribution, an improved estimate for global Gaussian Approximation can be obtained. Our contribution have been listed in the following subsection \ref{cont}.\\



\section{Contribution}
\label{cont}
Our main contribution could be concluded as the following part:
\begin{itemize}
	\item Introduction of Lasso Distribution
	\item Derivation of properties for Univariate Lasso Distribution.
	\item Derivation of properties for Multivariate Lasso Distribution.
	\item Implementation of Univariate Lasso Distribution and Multivariate Lasso Distribution property in R.
	\item Design of two new Variational Inference approaches based on local approximation by univariate lasso distribution and multivariate lasso distribution respectively.
	\item Conduct of experiment to testify two algorithms under  dataset by several evaluation metrics for approximation accuracy such as .
\end{itemize}



\section{Thesis Organization}
This paper will be divided up into 5 chapters. Chapter 1 briefly illustrate the motivation and background of the Lasso problem, Bayesian Lasso Problem and fast Approximate Bayesian Inference such as Variational Approximation. Chapter 2 will briefly review and explain the details of the methodlogy in previous work such as the lasso problem, MCMC(Monte Carlo Method) and their variants and Mean-Field Variational Bayes(MFVB). We will present our main methodlogy of variational correction algorithm in Chapter 3, followed by a comprehensive experiment for testing the effectiveness of algorithm in Chapter 4. In Chapter 5, we will briefly discuss and explain our result and potential improvement in the future. 



 