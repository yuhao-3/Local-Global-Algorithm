\usepackage[utf8]{inputenc}
\usepackage[dvipsnames]{xcolor}
\usepackage{graphicx}
\usepackage{bm}
\usepackage{inputenc}
\usepackage{xfrac} % for the slanted fraction command
\usepackage{caption}
\usepackage{subcaption}
\usepackage{float}
\usepackage{array} % for alignment of tabular when fixing col width
\usepackage{multirow} % multirows in tabular
\usepackage{titling, titlesec} % for titling options / sections etc.



% Hyperlinks for TOCs, and general link settings
\usepackage{hyperref}
\hypersetup{
    colorlinks,
    citecolor={blue},
    filecolor=black,
    linkcolor=blue,
    urlcolor={blue}
}

% put text above symbols
\usepackage{mathtools}
\newcommand\myeq{\stackrel{\mathclap{\normalfont\mbox{s}}}{~}}
\newcommand{\widesim}[2][1.5]{
  \mathrel{\overset{#2}{\scalebox{#1}[1]{$\sim$}}}
}


\usepackage[margin=1in]{geometry} % changing document margins
%\usepackage{hyperref} % hyperlinks for URLs
 
%\setlength\parindent{0pt} % For no indenting
\linespread{1.6}

% -------- Title page

\numberwithin{equation}{chapter}
\numberwithin{section}{chapter}
%\numberwithin{subsection}{chapter}


%References
% \usepackage[numbers]{natbib}
% \bibliographystyle{unsrtnat}   
% \usepackage[nottoc]{tocbibind}


%Page Formatting
\usepackage[titletoc]{appendix}
\usepackage[english]{babel}
%\usepackage[none]{hyphenat}

\usepackage{geometry}
 \geometry{
 a4paper,
 total={170mm,242.5mm},
 left=22mm,
 top=25mm,
 }

\titleformat{\chapter}[display]{\normalfont\huge\bfseries\centering}   {\chaptertitlename\ \thechapter}{20pt}{\Huge}
\titlespacing{\chapter}{0pt}{-32pt}{1cm}

\titleformat*{\section}{\normalfont \fontsize{16}{19.2} \bfseries}  

% array package and thick rules for tables
\usepackage{array}

% create "+" rule type for thick vertical lines
\newcolumntype{+}{!{\vrule width 2pt}}

% create \thickcline for thick horizontal lines of variable length
\newlength\savedwidth
\newcommand\thickcline[1]{%
  \noalign{\global\savedwidth\arrayrulewidth\global\arrayrulewidth 2pt}%
  \cline{#1}%
  \noalign{\vskip\arrayrulewidth}%
  \noalign{\global\arrayrulewidth\savedwidth}%
}

% \thickhline command for thick horizontal lines that span the table
\newcommand\thickhline{\noalign{\global\savedwidth\arrayrulewidth\global\arrayrulewidth 2pt}%
\hline
\noalign{\global\arrayrulewidth\savedwidth}}