\documentclass[12pt]{article}

\usepackage[utf8]{inputenc}



\pdfoutput=1

\usepackage{amsmath,amssymb,amsfonts,color}
\usepackage{graphicx,psfrag,epsf}
\usepackage{enumerate}
\usepackage[authoryear]{natbib}

\usepackage{float}

\usepackage{algorithm,algorithmic}

\usepackage{url} % not crucial - just used below for the URL 

%\pdfminorversion=4
% NOTE: To produce blinded version, replace "0" with "1" below.
\newcommand{\blind}{1}

% DON'T change margins - should be 1 inch all around.
 

\usepackage{geometry}
%\geometry{verbose,a4paper,tmargin=1cm,bmargin=1.0cm,lmargin=1cm,rmargin=1cm}
\geometry{verbose,a4paper,tmargin=1in,bmargin=1.0in,lmargin=1in,rmargin=1in}



% special symbols and abbreviations
% Sets or statistical values
\def\sI{{\mathcal I}}                            % Current Index set
\def\mJ{{\mathcal J}}                            % Select Index set
\def\sL{{\mathcal L}}                            % Likelihood
\def\sl{{\ell}}                                  % Log-likelihood
\def\sN{{\mathcal N}}
\def\sS{{\mathcal S}}
\def\sP{{\mathcal P}}
\def\sQ{{\mathcal Q}}
\def\sB{{\mathcal B}}
\def\sD{{\mathcal D}}
\def\sT{{\mathcal T}}
\def\sE{{\mathcal E}}
\def\sF{{\mathcal F}}
\def\sC{{\mathcal C}}
\def\sO{{\mathcal O}}
\def\sH{{\mathcal H}}
\def\sR{{\mathcal R}}
\def\mJ{{\mathcal J}}
\def\sCP{{\mathcal CP}}
\def\sX{{\mathcal X}}
\def\sA{{\mathcal A}}
\def\sZ{{\mathcal Z}}
\def\sM{{\mathcal M}}
\def\sK{{\mathcal K}}
\def\sG{{\mathcal G}}
\def\sY{{\mathcal Y}}
\def\sU{{\mathcal U}}


\def\sIG{{\mathcal IG}}


\def\cD{{\sf D}}
\def\cH{{\sf H}}
\def\cI{{\sf I}}

% Vectors
\def\vectorfontone{\bf}
\def\vectorfonttwo{\boldsymbol}
\def\va{{\vectorfontone a}}                      %
\def\vb{{\vectorfontone b}}                      %
\def\vc{{\vectorfontone c}}                      %
\def\vd{{\vectorfontone d}}                      %
\def\ve{{\vectorfontone e}}                      %
\def\vf{{\vectorfontone f}}                      %
\def\vg{{\vectorfontone g}}                      %
\def\vh{{\vectorfontone h}}                      %
\def\vi{{\vectorfontone i}}                      %
\def\vj{{\vectorfontone j}}                      %
\def\vk{{\vectorfontone k}}                      %
\def\vl{{\vectorfontone l}}                      %
\def\vm{{\vectorfontone m}}                      % number of basis functions
\def\vn{{\vectorfontone n}}                      % number of training samples
\def\vo{{\vectorfontone o}}                      %
\def\vp{{\vectorfontone p}}                      % number of unpenalized coefficients
\def\vq{{\vectorfontone q}}                      % number of penalized coefficients
\def\vr{{\vectorfontone r}}                      %
\def\vs{{\vectorfontone s}}                      %
\def\vt{{\vectorfontone t}}                      %
\def\vu{{\vectorfontone u}}                      % Penalized coefficients
\def\vv{{\vectorfontone v}}                      %
\def\vw{{\vectorfontone w}}                      %
\def\vx{{\vectorfontone x}}                      % Covariates/Predictors
\def\vy{{\vectorfontone y}}                      % Targets/Labels
\def\vz{{\vectorfontone z}}                      %

\def\vone{{\vectorfontone 1}}
\def\vzero{{\vectorfontone 0}}

\def\valpha{{\vectorfonttwo \alpha}}             %
\def\vbeta{{\vectorfonttwo \beta}}               % Unpenalized coefficients
\def\vgamma{{\vectorfonttwo \gamma}}             %
\def\vdelta{{\vectorfonttwo \delta}}             %
\def\vepsilon{{\vectorfonttwo \epsilon}}         %
\def\vvarepsilon{{\vectorfonttwo \varepsilon}}   % Vector of errors
\def\vzeta{{\vectorfonttwo \zeta}}               %
\def\veta{{\vectorfonttwo \eta}}                 % Vector of natural parameters
\def\vtheta{{\vectorfonttwo \theta}}             % Vector of combined coefficients
\def\vvartheta{{\vectorfonttwo \vartheta}}       %
\def\viota{{\vectorfonttwo \iota}}               %
\def\vkappa{{\vectorfonttwo \kappa}}             %
\def\vlambda{{\vectorfonttwo \lambda}}           % Vector of smoothing parameters
\def\vmu{{\vectorfonttwo \mu}}                   % Vector of means
\def\vnu{{\vectorfonttwo \nu}}                   %
\def\vxi{{\vectorfonttwo \xi}}                   %
\def\vpi{{\vectorfonttwo \pi}}                   %
\def\vvarpi{{\vectorfonttwo \varpi}}             %
\def\vrho{{\vectorfonttwo \rho}}                 %
\def\vvarrho{{\vectorfonttwo \varrho}}           %
\def\vsigma{{\vectorfonttwo \sigma}}             %
\def\vvarsigma{{\vectorfonttwo \varsigma}}       %
\def\vtau{{\vectorfonttwo \tau}}                 %
\def\vupsilon{{\vectorfonttwo \upsilon}}         %
\def\vphi{{\vectorfonttwo \phi}}                 %
\def\vvarphi{{\vectorfonttwo \varphi}}           %
\def\vchi{{\vectorfonttwo \chi}}                 %
\def\vpsi{{\vectorfonttwo \psi}}                 %
\def\vomega{{\vectorfonttwo \omega}}             %


% Matrices
%\def\matrixfontone{\sf}
%\def\matrixfonttwo{\sf}
\def\matrixfontone{\bf}
\def\matrixfonttwo{\boldsymbol}
\def\mA{{\matrixfontone A}}                      %
\def\mB{{\matrixfontone B}}                      %
\def\mC{{\matrixfontone C}}                      % Combined Design Matrix
\def\mD{{\matrixfontone D}}                      % Penalty Matrix for \vu_J
\def\mE{{\matrixfontone E}}                      %
\def\mF{{\matrixfontone F}}                      %
\def\mG{{\matrixfontone G}}                      % Penalty Matrix for \vu
\def\mH{{\matrixfontone H}}                      %
\def\mI{{\matrixfontone I}}                      % Identity Matrix
\def\mJ{{\matrixfontone J}}                      %
\def\mK{{\matrixfontone K}}                      %
\def\mL{{\matrixfontone L}}                      % Lower bound
\def\mM{{\matrixfontone M}}                      %
\def\mN{{\matrixfontone N}}                      %
\def\mO{{\matrixfontone O}}                      %
\def\mP{{\matrixfontone P}}                      %
\def\mQ{{\matrixfontone Q}}                      %
\def\mR{{\matrixfontone R}}                      %
\def\mS{{\matrixfontone S}}                      %
\def\mT{{\matrixfontone T}}                      %
\def\mU{{\matrixfontone U}}                      % Upper bound
\def\mV{{\matrixfontone V}}                      %
\def\mW{{\matrixfontone W}}                      % Variance Matrix i.e. diag(b'')
\def\mX{{\matrixfontone X}}                      % Unpenalized Design Matrix/Nullspace Matrix
\def\mY{{\matrixfontone Y}}                      %
\def\mZ{{\matrixfontone Z}}                      % Penalized Design Matrix/Kernel Space Matrix

\def\mGamma{{\matrixfonttwo \Gamma}}             %
\def\mDelta{{\matrixfonttwo \Delta}}             %
\def\mTheta{{\matrixfonttwo \Theta}}             %
\def\mLambda{{\matrixfonttwo \Lambda}}           % Penalty Matrix for \vnu
\def\mXi{{\matrixfonttwo \Xi}}                   %
\def\mPi{{\matrixfonttwo \Pi}}                   %
\def\mSigma{{\matrixfonttwo \Sigma}}             %
\def\mUpsilon{{\matrixfonttwo \Upsilon}}         %
\def\mPhi{{\matrixfonttwo \Phi}}                 %
\def\mOmega{{\matrixfonttwo \Omega}}             %
\def\mPsi{{\matrixfonttwo \Psi}}                 %

\def\mone{{\matrixfontone 1}}
\def\mzero{{\matrixfontone 0}}

% Fields or Statistical
\def\bE{{\mathbb E}}                             % Expectation
\def\bC{{\mathbb C}}                             % Expectation
\def\bS{{\mathbb S}}                             % Expectation
\def\bP{{\mathbb P}}                             % Probability
\def\bR{{\mathbb R}}                             % Reals
\def\bI{{\mathbb I}}                             % Reals
\def\bV{{\mathbb V}}                             % Reals
\def\bN{{\mathbb N}}
\def\bK{{\mathbb K}}
\def\bM{{\mathbb M}}
\def\bW{{\mathbb W}}
 

\def\d{\partial}
\def\ds{\displaystyle}
 
\def\given{\,|\,}


\def\argmax{\operatornamewithlimits{\text{argmax}}}

\newtheorem{result}{Result}

\newtheorem{proof}{Proof}

\definecolor{orange}{rgb}{1,0.5,0}
\definecolor{purple}{rgb}{0.75,0,1}
\definecolor{darkgreen}{rgb}{0,0.5,0}

\newcommand{\joc}[1]{{\color{blue}#1}}
\newcommand{\src}[1]{{\color{red}#1}}
\newcommand{\wyc}[1]{{\color{darkgreen}#1}}
\newcommand{\msc}[1]{{\color{purple}#1}}
\usepackage{rotating}

\def\grad{\nabla}

\title{Local Global Approximations (working title)}

\author{John T. Ormerod}
\date{\today}


\usepackage{geometry}
\geometry{verbose,a4paper,tmargin=1in,bmargin=1.0in,lmargin=1in,rmargin=1in}

\usepackage{atkinson}
 


\begin{document}

\maketitle

\section{Introduction}

We will continue to work on the Lasso becuse that is a good place to start.

 

\section{The Lasso Distribution}

If $x \sim \mbox{Lasso}(a,b,c)$ with then it has density given by
$$
p(x,a,b,c) = Z^{-1}\exp\left( -\tfrac{1}{2}ax^2 + bx - c|x| \right)
$$

\noindent where $x\in\bR$, $a>0$, $b\in\bR$, $c>0$, and $Z$ is the normalizing constant. Then
$$
\begin{array}{rl}
Z(a,b,c)
& \ds = \int_{-\infty}^\infty \exp\left[ -\tfrac{1}{2}ax^2 + bx - c|x| \right] dx
\\ [2ex]
& \ds 
= \int_0^\infty    \exp\left[ -\tfrac{1}{2}ax^2 + (b - c)x \right] dx
+ \int_{-\infty}^0 \exp\left[ -\tfrac{1}{2}ax^2 + (b + c)x \right] dx
\\ [2ex]
& \ds 
= \int_0^\infty \exp\left[ -\tfrac{1}{2}ax^2 + (b - c)x \right] dx
+ \int_0^\infty \exp\left[ -\tfrac{1}{2}ay^2 - (b + c)y \right] dy
\\ [2ex]
& \ds 
= \int_0^\infty \exp\left[ - \frac{(x - \mu_1)^2}{2\sigma^2} + \frac{\mu_1^2}{2\sigma^2} \right] dx
+ \int_0^\infty \exp\left[ - \frac{(x - \mu_2)^2}{2\sigma^2} + \frac{\mu_2^2}{2\sigma^2} \right] dy
\\ [2ex]
& \ds 
= \sqrt{2\pi\sigma^2}
\left[  \exp\left\{  \frac{\mu_1^2}{2\sigma^2} \right\} \int_0^\infty \phi(x;\mu_1,\sigma^2) dx
+       \exp\left\{  \frac{\mu_2^2}{2\sigma^2} \right\} \int_0^\infty \phi(y;\mu_2,\sigma^2) dy
\right] 
\\ [2ex]
& \ds 
= \sqrt{2\pi\sigma^2}
\left[  \exp\left\{  \frac{\mu_1^2}{2\sigma^2} \right\} \left\{ 1 - \Phi(-\mu_1/\sigma) \right\} 
+       \exp\left\{  \frac{\mu_2^2}{2\sigma^2} \right\} \left\{ 1 - \Phi(-\mu_2/\sigma) \right\} 
\right] 
\\ [2ex]
& \ds 
= \sqrt{2\pi\sigma^2}
\left[  \exp\left(  \frac{\mu_1^2}{2\sigma^2} \right) \Phi\left(\frac{\mu_1}{\sigma} \right) 
+       \exp\left(  \frac{\mu_2^2}{2\sigma^2} \right) \Phi\left( \frac{\mu_2}{\sigma} \right)  
\right] 


\\ [2ex]
& \ds 
= 
  \sigma \left[ \frac{\Phi(\mu_1/\sigma)}{\phi(\mu_1/\sigma)}
+ \frac{\Phi(\mu_2/\sigma)}{\phi(\mu_2/\sigma)}  \right] 


 
\end{array} 
$$

\noindent where $\mu_1 = (b-c)/a$, $\mu_2 = -(c + b)/a$ and $\sigma^2 = 1/a$. Care should be taken when evaluating $Z$ (which is prone to overflow and divide by zero problems) and is a function of the Mills ratio. 

\subsection{The Moment Generating function and Moments}


\noindent The moment generating function requires almost identical calculations with $b$ replaced with $b+t$.
$$
M(t) = \frac{Z(a,b+t,c)}{Z(a,b,c) }
$$

\noindent While this is true it doesn't look useful for calculating moments.

\subsection{Moments}

The moments of the Lasso distribution are:
$$
\begin{array}{rl}
E(x^r)
& \ds = Z^{-1} \int_{-\infty}^\infty x^r \exp\left[ -\tfrac{1}{2}ax^2 + bx - c|x| \right] dx
\\ [2ex]
& \ds 
= Z^{-1}  \int_0^\infty   x^r \exp\left[ -\tfrac{1}{2}ax^2 + (b - c)x \right] dx
+ \int_{-\infty}^0 x^r \exp\left[ -\tfrac{1}{2}ax^2 + (b + c)x \right] dx
\\ [2ex]
& \ds 
=  Z^{-1}  \int_0^\infty x^r \exp\left[ -\tfrac{1}{2}ax^2 + (b - c)x \right] dx
+ (-1)^r\int_0^\infty y^r \exp\left[ -\tfrac{1}{2}ay^2 - (b + c)y \right] dy
\\ [2ex]
& \ds 
= Z^{-1}  \sqrt{2\pi\sigma^2}
 \exp\left(  \frac{\mu_1^2}{2\sigma^2} \right) \int_0^\infty x^r \phi(x;\mu_1,\sigma^2) dx
\\ [2ex]
& \ds \qquad + (-1)^r    \sqrt{2\pi\sigma^2}   \exp\left(  \frac{\mu_2^2}{2\sigma^2} \right) \int_0^\infty y^r \phi(y;\mu_2,\sigma^2) dy

\\ [2ex]
& \ds 
= \frac{\sigma}{Z} \left[  
 \frac{\Phi(\mu_1/\sigma)}{\phi(\mu_1/\sigma)} \frac{\int_0^\infty x^r \phi(x;\mu_1,\sigma^2) dx}{\Phi(\mu_1/\sigma)}
+ (-1)^r  \frac{\Phi(\mu_2/\sigma)}{\phi(\mu_2/\sigma)}  \frac{\int_0^\infty y^r \phi(y;\mu_2,\sigma^2) dy}{\Phi(\mu_2/\sigma)}
\right] 

\\ [4ex]
& \ds 
= \frac{\sigma}{Z} \left[  
 \frac{\Phi(\mu_1/\sigma)}{\phi(\mu_1/\sigma)} 
 \bE( A^r )
+ (-1)^r  \frac{\Phi(\mu_2/\sigma)}{\phi(\mu_2/\sigma)}  \bE( B^r )
\right] 
\end{array} 
$$

\noindent where $A\sim TN_+(\mu_1,\sigma^2)$, $B\sim TN_+(\mu_2,\sigma^2)$ and $TN_+$ is denotes the positively truncated normal distribution.
Note that
$$
\bE(A) = \mu_1 + \frac{\sigma \phi(\mu_1/\sigma)}{\Phi(\mu_1/\sigma)} = \mu_1 + \sigma \zeta_1(\mu_1/\sigma)
$$

\noindent and
$$
\bV(A) = \sigma^2  \left[ 1 + \zeta_2(\mu_1/\sigma) \right] 
$$

\noindent where $\zeta_k(x) = d^k \log \Phi(x)/dx^k$,
$\zeta_1(t) = \phi(t)/\Phi(t)$, $\zeta_2(t) = -t\,\zeta_1(t) - \zeta_1(t)^2$.
Here
$\zeta_1(x)$ is the inverse Mills ratio which too needs to be treated with care.
Hence,
$$
\bE(A^2) = \bV(A) + \bE(A)^2 = \sigma^2  \left[ 1 + \zeta_2(\mu_1/\sigma) \right] + \left[\mu_1 + \sigma \zeta_1(\mu_1/\sigma) \right]^2
$$

\noindent We now have sufficient information to calculate the moments of the Lasso distribution.
We also have sufficient information to implement a VB approximation.

\subsection{CDF}



\noindent 
Similarly if
$z\le 0$ the CDF is given by
$$
\begin{array}{rl} 
P(Z < z) 
& \ds = Z^{-1} \int_{-\infty}^z \exp\left[ -\tfrac{1}{2}ax^2 + (b + c)x \right] dx 
\\ [2ex]
& \ds = Z^{-1} \sqrt{2\pi\sigma^2} \exp\left( \frac{\mu_2^2}{2\sigma^2} \right) \int_{-\infty}^z \phi(x;-\mu_2,\sigma^2)
\\ [2ex]
& \ds = Z^{-1} \sqrt{2\pi\sigma^2} \exp\left( \frac{\mu_2^2}{2\sigma^2} \right) \Phi\left( \frac{z + \mu_2}{\sigma} \right)
\\ [2ex]
& \ds =  \frac{\sigma}{Z}\frac{\ds \Phi\left( \frac{z + \mu_2}{\sigma} \right)}{\phi(\mu_2/\sigma)}
\end{array} 
$$

\noindent and if $z>0$ we have
$$
\begin{array}{rl}
P(Z < z)
& \ds = Z^{-1}\int_{-\infty}^z \exp\left( -\tfrac{1}{2}ax^2 + bx - c|x| \right) dx
\\ [2ex]
& \ds 
= Z^{-1} \sqrt{2\pi\sigma^2}
\left[  \exp\left(  \frac{\mu_1^2}{2\sigma^2} \right) \int_0^z \phi(x;\mu_1,\sigma^2) dx
+      \exp\left(  \frac{\mu_2^2}{2\sigma^2} \right) \Phi\left( \frac{\mu_2}{\sigma} \right)  
\right] 
\\ [2ex]
& \ds = \frac{\sigma}{Z} 
\left[  \frac{ 
\Phi\left( \frac{z - \mu_1}{\sigma} \right) - \Phi\left( \frac{ - \mu_1}{\sigma} \right)}{\phi(\mu_1/\sigma)}
+       \frac{\Phi\left( \mu_2/\sigma \right)}{\phi(\mu_2/\sigma)}  
\right] 

\end{array} 
$$


\subsection{Inverse CDF}

\noindent For the inverse CDF we again have two cases. Let $u = P(Z<z)$. When 
$$
u \le \frac{\sigma}{Z}  \frac{\ds \Phi\left(   \mu_2/\sigma \right)}{\phi(\mu_2/\sigma)}
$$ 

\noindent we solve
$$
u = \frac{\ds \sigma \Phi\left( \frac{z + \mu_2}{\sigma} \right)}{Z \phi(\mu_2/\sigma)}
$$

\noindent for $z$ to obtain
$$
z = \mu_2 + \sigma \,\Phi^{-1}\left[ (Z/\sigma) \phi_\sigma(\mu_2/\sigma) u \right]
$$

\noindent 
When 
$$
u > \frac{\sigma}{Z}  \frac{\ds \Phi\left(   \mu_2/\sigma \right)}{\phi(\mu_2/\sigma)}
$$ 

\noindent 
we need to solve
$$
u = \frac{\sigma}{Z} 
\left[  \frac{ 
\Phi\left( \frac{z - \mu_1}{\sigma} \right) - \Phi\left( \frac{ - \mu_1}{\sigma} \right)}{\phi(\mu_1/\sigma)}
+       \frac{\Phi\left( \mu_2/\sigma \right)}{\phi(\mu_2/\sigma)}  
\right] 
$$

\noindent for $z$ to obtain
$$
z = \mu_1 + \sigma\, \Phi^{-1}\left[ \phi(\mu_1/\sigma) \left\{ \frac{Z u}{\sigma} 
-       \frac{\Phi\left( \mu_2/\sigma \right)}{\phi(\mu_2/\sigma)} \right\} 
+ \Phi\left( \frac{ - \mu_1}{\sigma} \right) \right] 
$$

\noindent which also involves the Mills ratio.



\newpage 

\section{Tasks}

Suppose that conformably we have
$$
\vmu = \left[ \begin{array}{c} 
\vmu_1 \\
\vmu_2 
\end{array} \right] \qquad \mbox{and} \qquad 
\mSigma = \left[ \begin{array}{cc} 
\mSigma_{11} & \mSigma_{12} \\
\mSigma_{21} & \mSigma_{22}
\end{array} \right]
$$
    
\noindent and $\vtheta = (\vtheta_1,\vtheta_2)$ and
$q(\vtheta)\sim N(\vmu,\mSigma)$ approximates $p(\vtheta\mid\sD)$. 
    


\begin{enumerate}
    \item Verify the algebra in Section 2 above for the Lasso distribution.
    
        
    \item Show that
    $$
    p(\vtheta_1\mid\sD) = \int p(\vtheta_1\mid\sD,\vtheta_2)p(\vtheta_2\mid\sD) d\vtheta_2
    $$
    
    \item Find $q(\vtheta_2\mid\vtheta_1)$.
    
        
    {\bf Solution:} The conditional distribution of $\vtheta_1|\vtheta_2$ is
    $$
    \phi\left(
    \vtheta_2;
\boldsymbol\mu_2 + \boldsymbol\Sigma_{21} \boldsymbol\Sigma_{11}^{-1}
\left(
 \vtheta_1 - \boldsymbol\mu_1
\right),
{\boldsymbol {\Sigma }}_{22}-{\boldsymbol {\Sigma }}_{21}{\boldsymbol {\Sigma }}_{11}^{-1}{\boldsymbol {\Sigma }}_{12}\right)
    $$
    
    \item Suppose that we approximate $p(\vtheta_2\mid\sD)$ by $q(\vtheta_2) = N(\vmu_2,\mSigma_{22})$. Suppose we use this to approximate 
    $p(\vtheta_1\mid\sD)$ by
    $$
    q^*(\vtheta_1) = \int p(\vtheta_1\mid\sD,\vtheta_2)q(\vtheta_2)  d\vtheta_2
    $$
    
    \noindent Suppose that the mean of $q^*(\vtheta_1)$ is $\vmu_1^*$, covariance is $\mSigma_{11}^*$, and $q^*(\vtheta_1) \approx N(\vmu_1^*,\mSigma_{11}^*)$.
    We update $q(\vtheta)$ via
    $$
    q^*(\vtheta) = q(\vtheta_2\mid\vtheta_1) \phi(\vtheta_1;\vmu_1^*,\mSigma_{11}^*)
    $$
    
    \noindent Find $q^*(\vtheta)$.
    
    \newpage 
    
    {\bf Solution [The easy way]:} We know that $q(\vtheta_2|\vtheta_1)$ and $q(\vtheta_1)$ are Gaussian. Hence, their joint distribution will also be Gaussian. The marginal 
    mean and variance of the joint distribution for $\vtheta_1$ will be $\vmu_1^*$
    and $\mSigma_{11}^*$. Suppose the mean and covariance of the joint distribution
    are $\widetilde{\vmu}$ and $\widetilde{\mSigma}$ respectively. Then
    $\widetilde{\vmu}_1 = \vmu_1^*$, and $\widetilde{\mSigma}_{11} = \mSigma_{11}^*$.
    All we need to do is determine $\widetilde{\vmu}_{2}$, $\widetilde{\mSigma}_{22}$
    and $\widetilde{\mSigma}_{12}$.
    
    \bigskip 
    \noindent We have
    $$
    \begin{array}{rl}
    \bE(\vtheta_2) 
    & \ds = \bE[\bE(\vtheta_2\mid\vtheta_1)] 
    \\ [1ex]
    & \ds = \mE[\vmu_2 + \mSigma_{21}\mSigma_{11}^{-1}\left(\vtheta_1 - \vmu_1\right)]
    \\ [1ex]
    & \ds = \vmu_2 + \mSigma_{21}\mSigma_{11}^{-1}\left(\vmu_1^* - \vmu_1\right).
    \end{array} 
    $$
    
    \noindent Similarly,
    $$
    \begin{array}{rl}
    \mbox{Cov}(\vtheta_2) 
    & \ds = \bE[\mbox{Cov}(\vtheta_2\mid\vtheta_1)] + \mbox{Cov}[\bE(\vtheta_2\mid\vtheta_1)] 
    \\ [2ex]
    & \ds = \bE(\mSigma_{22} - \mSigma_{21} \mSigma_{11}^{-1}\mSigma_{12}) + \mbox{Cov}[\vmu_2 + \mSigma_{21}\mSigma_{11}^{-1}\left(\vtheta_1 - \vmu_1\right)] 
    \\ [2ex]
    & \ds = \mSigma_{22} - \mSigma_{21} \mSigma_{11}^{-1}\mSigma_{12}
    + \mSigma_{21}\mSigma_{11}^{-1}  \mbox{Cov}(\vtheta_1) \mSigma_{11}^{-1}  \mSigma_{12}
    \\ [2ex]
    & \ds =  \mSigma_{22} 
    + \mSigma_{21}(\mSigma_{11}^{-1}  \mSigma_{11}^* \mSigma_{11}^{-1} -\mSigma_{11}^{-1})  \mSigma_{12}.
    \end{array} 
    $$
    
    \noindent Lastly,
    $$
    \begin{array}{rl}
    \mbox{Cov}(\vtheta_1,\vtheta_2) 
    & \ds = \bE[(\vtheta_1 - \bE(\vtheta_1))(\vtheta_2 - \bE(\vtheta_2))^T] 
    \\ [2ex]
    & \ds = \bE\left[ \bE\left\{ (\vtheta_1 - \bE(\vtheta_1))(\vtheta_2 - \bE(\vtheta_2))^T \mid \vtheta_1 \right\} \right]     \\ [2ex]
    & \ds = \bE\left[ \left(\vtheta_1 - \vmu_1^* \right)\left(  \mSigma_{21}\mSigma_{11}^{-1}(\vtheta_1 - \vmu_1^*)    \right)^T\right] 
    \\ [2ex]
    & \ds = \bE\left[ \left(\vtheta_1 - \vmu_1^* \right)\left(  \vtheta_1 - \vmu_1^* \right)^T\right] \mSigma_{11}^{-1}\mSigma_{12}
    \\ [2ex]
    & \ds = \mSigma_{11}^* \mSigma_{11}^{-1}\mSigma_{12}.
    \end{array} 
    $$
    
    \noindent Hence, $q^*(\vtheta) = N(\widetilde{\vmu},\widetilde{\mSigma})$ where
    $$
    \widetilde{\vmu} =
    \left[ \begin{array}{c}
    \vmu_1^* \\
     \vmu_2 + \mSigma_{21}\mSigma_{11}^{-1}\left(\vmu_1^* - \vmu_1\right)
    \end{array} \right]
    $$
    
    \noindent and
    $$
    \widetilde{\mSigma} = 
    \left[ \begin{array}{cc}
    \mSigma_{11}^* & \mSigma_{11}^* \mSigma_{11}^{-1}\mSigma_{12} \\
    \mSigma_{21}  \mSigma_{11}^{-1}\mSigma_{11}^* & \mSigma_{22} 
    + \mSigma_{21}\mSigma_{11}^{-1}  ( \mSigma_{11}^* -\mSigma_{11})  \mSigma_{11}^{-1} \mSigma_{12}
    \end{array} \right].
    $$
    
    \noindent Note that
    $$
    \widetilde{\mOmega} = \widetilde{\mSigma}^{-1} = \left[ \begin{array}{cc}
    (\mSigma_{11}^*)^{-1}  + \mSigma_{11}^{-1} \mSigma_{12} \mQ \mSigma_{21} \mSigma_{11}^{-1}
        & - \mQ \mSigma_{21}\mSigma_{11}^{-1} 
    \\ 
    - \mSigma_{11}^{-1} \mSigma_{12} \mQ
    & \mQ
    \end{array} \right] 
    $$
    
    \noindent where
    $$
    \mQ =  (\mSigma_{22} - \mSigma_{21} \mSigma_{11}^{-1}\mSigma_{22})^{-1}.
    $$
    
    \noindent Note that only the top right hand block of $\widetilde{\mOmega}$ changes when $\mSigma_{11}^*$ changes. This might be useful because $\widetilde{\mOmega}$ will be approximately sparse. This might be important for high dimensional problems.
    
    
    
    \noindent A matrix 
    $\widetilde{\mSigma}$ is positive definite if and only if the upper left block and its Schur complement are positive definite (see Theorem 7.7.6  of Horn \& Johnson, 2012). The upper block is $\mSigma_{11}^*$ which we will assume is positive definite. The Schur complement is given by
    $$
    \begin{array}{l}
    \mSigma_{22} + \mSigma_{21}\mSigma_{11}^{-1}  ( \mSigma_{11}^* -\mSigma_{11})  \mSigma_{11}^{-1} \mSigma_{12} - \mSigma_{21}  \mSigma_{11}^{-1}\mSigma_{11}^* (\mSigma_{11}^*)^{-1} \mSigma_{11}^* \mSigma_{11}^{-1}\mSigma_{12}
    \\ [2ex]
    \qquad \ds  = \mSigma_{22} + \mSigma_{21}\mSigma_{11}^{-1}  ( \mSigma_{11}^* -\mSigma_{11})  \mSigma_{11}^{-1} \mSigma_{12} - \mSigma_{21}  \mSigma_{11}^{-1}\mSigma_{11}^*  \mSigma_{11}^{-1}\mSigma_{12}
    \\ [2ex]
    \qquad  \ds = \mSigma_{22} - \mSigma_{21}\mSigma_{11}^{-1} \mSigma_{12}  
    \end{array} 
    $$
    
    \noindent which is positive definite since it is the Schur complement of a positive definite matrix.
    
    \newpage 
    
    {\bf Solution [The hard way]:}
    \noindent Consider,
    $$
    \begin{array}{l}
    q(\vtheta_2\mid\vtheta_1) \phi(\vtheta_1;\vmu_1^*,\mSigma_{11}^*)
    \\ [2ex] 
    \quad \ds \propto \exp\left[ 
    -\tfrac{1}{2}\left(
    \vtheta_2 - \vmu_2 - \mSigma_{21}\mSigma_{11}^{-1}\left(\vtheta_1 - \vmu_1\right)\right)^T \mQ
    \left( \vtheta_2 - \vmu_2 - \mSigma_{21}\mSigma_{11}^{-1}\left(\vtheta_1 - \vmu_1\right)
    \right)
    \right]
    \\ [2ex]
    \quad  \ds \quad \times 
    \exp\left[ 
    -\tfrac{1}{2} (\vtheta_1 - \vmu_1^*)^T(\mSigma_{11}^*)^{-1}(\vtheta_1 - \vmu_1^*)
    \right] 
    
    
    
    \\ [2ex]
    
    
    
    \quad \ds = \exp\left[ 
    -\tfrac{1}{2}
    \vtheta_2^T \mQ \vtheta_2 +\left( \vmu_2 - \mSigma_{21}\mSigma_{11}^{-1}\vmu_1\right)^T  \mQ\vtheta_2
    
    \right]
    \\ [2ex]
    \quad \ds  \quad \times \exp\left[ 
    -\tfrac{1}{2}\vtheta_1^T\left( (\mSigma_{11}^*)^{-1}  + \mSigma_{11}^{-1} \mSigma_{12} \mQ\mSigma_{21}\mSigma_{11}^{-1} \right)\vtheta_1  
    \right]
    \\ [2ex]
    \quad \ds  \quad \times  \exp\left[ 
    \left( (\mSigma_{11}^*)^{-1} \vmu_1^* + \mSigma_{11}^{-1}\mSigma_{12}\mQ \left(  
    \vmu_2 - \mSigma_{21}\mSigma_{11}^{-1}\vmu_1\right)  \right)^T\vtheta_1
    \right]
    \\ [2ex]
    \quad \ds  \quad \times  \exp\left[
    \vtheta_2^T \mQ \mSigma_{21}\mSigma_{11}^{-1}\vtheta_1
    \right]
    
    \end{array} 
    $$
    
    \noindent where $\mQ = (\mSigma_{22} - \mSigma_{21} \mSigma_{11}^{-1}\mSigma_{22})^{-1}$
    
    \noindent Now suppose that $q^*(\vtheta)= N(\widetilde{\vmu},\widetilde{\mSigma})$ then
    $$
    \begin{array}{rl}
    q(\vtheta) 
    & \ds \propto \exp\left[ 
    - \frac{1}{2}\left( \left( \begin{array}{c} \vtheta_1 \\ \vtheta_2\end{array} \right) - \left( \begin{array}{c} \widetilde{\vmu}_1 \\ \vmu_2\end{array} \right)  \right)^T\left( \begin{array}{cc} \widetilde{\mQ}_{11} & \widetilde{\mQ}_{12} \\ \widetilde{\mQ}_{21} & \widetilde{\mQ}_{22}\end{array} \right)  \left( \left( \begin{array}{c} \vtheta_1 \\ \vtheta_2\end{array} \right) - \left( \begin{array}{c} \widetilde{\vmu}_1 \\ \widetilde{\vmu}_2\end{array} \right)  \right)
    \right] 
    \\ [3ex]
    & \ds \propto 
    \exp\left[ - \tfrac{1}{2}(\vtheta_1 - \widetilde{\vmu}_1)^T\widetilde{\mQ}_{11}(\vtheta_1 - \widetilde{\vmu}_1) \right]
    \\ [2ex]
    & \ds \qquad \times \exp\left[ - \tfrac{1}{2}(\vtheta_2 - \widetilde{\vmu}_2)^T\widetilde{\mQ}_{22}(\vtheta_2 - \widetilde{\vmu}_2) \right]
        \\ [2ex]
    & \ds \qquad \times \exp\left[ - (\vtheta_1 - \widetilde{\vmu}_1)^T\widetilde{\mQ}_{12}(\vtheta_2 - \widetilde{\vmu}_2)
    \right] 
    \\ [3ex]
    & \ds \propto 
    \exp\left[ - \tfrac{1}{2}\vtheta_1^T\widetilde{\mQ}_{11}\vtheta_1 
    + \vtheta_1^T(\widetilde{\mQ}_{11}\widetilde{\vmu}_1 + \widetilde{\mQ}_{12}\widetilde{\vmu}_2) \right] 
            \\ [2ex]
    & \ds \qquad \times 
    \exp\left[ - \tfrac{1}{2}\vtheta_2^T\widetilde{\mQ}_{22}\vtheta_2 + \vtheta_2^T(\widetilde{\mQ}_{22}\widetilde{\vmu}_2 + \widetilde{\mQ}_{21}\widetilde{\vmu}_1)
    - \vtheta_2^T\widetilde{\mQ}_{21}\vtheta_1 \right] 
    \end{array} 
    $$
    
    \noindent where $\widetilde{\mQ} = \widetilde{\mSigma}^{-1}$. Matching term by term with the first expression above we find that
    $$
    \begin{array}{rl}
    \widetilde{\mQ}_{21} & \ds = -\mQ\mSigma_{21}\mSigma_{11}^{-1}
    \\ [2ex]
    \widetilde{\mQ}_{11} & \ds =  (\mSigma_{11}^*)^{-1}  + \mSigma_{11}^{-1} \mSigma_{12} \mQ\mSigma_{21}\mSigma_{11}^{-1} 
    \\ [2ex]
    \widetilde{\mQ}_{22} & \ds = \mQ
    \\ [2ex]
    \widetilde{\mQ}_{11}\widetilde{\vmu}_1 + \widetilde{\mQ}_{12}\widetilde{\vmu}_2
    & \ds =  
     (\mSigma_{11}^*)^{-1} \vmu_1^* + \mSigma_{11}^{-1}\mSigma_{12}\mQ \left(  
    \vmu_2 - \mSigma_{21}\mSigma_{11}^{-1}\vmu_1\right) 
    \\ [2ex]
    \widetilde{\mQ}_{22}\widetilde{\vmu}_2 + \widetilde{\mQ}_{21}\widetilde{\vmu}_1
    & \ds = \mQ\left( \vmu_2 - \mSigma_{21}\mSigma_{11}^{-1}\vmu_1\right) 
    \end{array} 
    $$
    
    \noindent We need to solve this system of equations to find
    $\widetilde{\mQ}_{11}$, $\widetilde{\mQ}_{12}$, $\widetilde{\mQ}_{22}$, $\widetilde{\vmu}_1$
    and $\widetilde{\vmu}_2$. We then need to use 
    $\widetilde{\mQ}_{11}$, $\widetilde{\mQ}_{12}$, and $\widetilde{\mQ}_{22}$
    to find
    $\widetilde{\mSigma}_{11}$, $\widetilde{\mSigma}_{12}$, $\widetilde{\mSigma}_{22}$.

    
    The block inverse formula states that the inverse of a real matrix can be written as
		\begin{eqnarray}
		\ds \left[ \begin{array}{cc}
		\mA   & \mB \\
		\mC & \mD
		\end{array} \right]^{-1}
		&  = &
		\ds \left[ \begin{array}{cc}
		\mI & \vzero \\
		-\mD^{-1}\mC &  \mI
		\end{array} \right]
		\left[ \begin{array}{cc}
		\widetilde{\mA} & \vzero \\
		\vzero & \mD^{-1}
		\end{array} \right]
		\left[ \begin{array}{cc}
		\mI    & -\mB\mD^{-1}\\
		\vzero & \mI
		\end{array} \right] \label{eq:blockdiag1}\\
		&  = &
		\ds\left[
		\begin{array}{cc}
		\widetilde{\mA}
		& - \widetilde{\mA}\mB\mD^{-1} \\
		-\mD^{-1}\mC\widetilde{\mA}
		& \mD^{-1} + \mD^{-1}\mC\widetilde{\mA}\mB\mD^{-1},
		\end{array}\right]\label{eq:blockdiag2}
		\end{eqnarray}
		
		\noindent where $\widetilde{\mA} = \left(\mA-\mB\mD^{-1}\mC\right)^{-1}$,
		provided all inverses in (\ref{eq:blockdiag1}) and
		(\ref{eq:blockdiag2}) exist.
    
    \noindent Using the block inverse formula for the top left element of $\mSigma$ we have
    $$
    \begin{array}{rl}
    \widetilde{\mSigma}_{11}  
    & \ds = \left( \widetilde{\mQ}_{11} - \widetilde{\mQ}_{12}\widetilde{\mQ}_{22}^{-1} \widetilde{\mQ}_{21} \right)^{-1}
    \\ [2ex]
    & \ds = \left( 
        (\mSigma_{11}^*)^{-1}  + \mSigma_{11}^{-1} \mSigma_{12} \mQ\mSigma_{21}\mSigma_{11}^{-1} 
        - \mSigma_{11}^{-1}\mSigma_{12}\mQ 
          \mQ^{-1} 
          \mQ\mSigma_{21}\mSigma_{11}^{-1}
    \right)^{-1}
    \\ [2ex]
    & \ds = \mSigma_{11}^* 
    \end{array} 
    $$
    
    \noindent as expected. Similarly, using the bottom right block inverse formula leads to
    $$
    \begin{array}{rl}
    \widetilde{\mSigma}_{22}
    & \ds = \mSigma_{22} + \mSigma_{21}(\mSigma_{11}^{-1}\mSigma_{11}^*\mSigma_{11}^{-1} - \mSigma_{11}^{-1}) \mSigma_{12}
    \end{array} 
    $$
    
    \noindent so that if $\mSigma_{11}^* = \mSigma_{11}$ then there will be no changes to $\mSigma_{22}$. 
    Finally, using the top left block inverse formula leads to 
    $$
    \begin{array}{rl}
    \widetilde{\mSigma}_{12} = \mSigma_{11}^*\mSigma_{11}^{-1} \mSigma_{12}
    \end{array} 
    $$
    
    \noindent Using the last equation in the system of equations to solve 
    $$
    \begin{array}{rl}
    \widetilde{\vmu}_2
    & \ds = \widetilde{\mQ}_{22}^{-1}\mQ\left( \vmu_2 - \mSigma_{21}\mSigma_{11}^{-1}\vmu_1\right)  - \widetilde{\mQ}_{22}^{-1}\widetilde{\mQ}_{21}\widetilde{\vmu}_1
    \\ [2ex]
    & \ds =  \vmu_2 + \mSigma_{21}\mSigma_{11}^{-1}(\widetilde{\vmu}_1 -  \vmu_1   )
    
    \end{array} 
    $$
    
    \noindent Using the 4th equation in the system of equations we have
    $$
    \begin{array}{rl}
    \widetilde{\mQ}_{11}\widetilde{\vmu}_1
    & \ds = (\mSigma_{11}^*)^{-1} \vmu_1^* + \mSigma_{11}^{-1}\mSigma_{12}\mQ \left(  
    \vmu_2 - \mSigma_{21}\mSigma_{11}^{-1}\vmu_1\right)
    -  \widetilde{\mQ}_{12}\widetilde{\vmu}_2
    
    \\ [2ex]
    & \ds = (\mSigma_{11}^*)^{-1} \vmu_1^* + \mSigma_{11}^{-1}\mSigma_{12}\mQ \left(  
    \vmu_2 - \mSigma_{21}\mSigma_{11}^{-1}\vmu_1\right)
    - \mSigma_{11}^{-1}\mSigma_{12}  \mQ \left( \vmu_2 + \mSigma_{21}\mSigma_{11}^{-1}(\widetilde{\vmu}_1 -  \vmu_1   ) \right)


    \\ [2ex]
    & \ds = (\mSigma_{11}^*)^{-1} \vmu_1^*
    - \mSigma_{11}^{-1}\mSigma_{12}  \mQ \mSigma_{21}\mSigma_{11}^{-1}\widetilde{\vmu}_1   
    
    
    
    \end{array} 
    $$
    
    \noindent Using the 2nd of the system of equations leads to 
    $$
    \left[ (\mSigma_{11}^*)^{-1}  + \mSigma_{11}^{-1} \mSigma_{12} \mQ\mSigma_{21}\mSigma_{11}^{-1} \right] \widetilde{\vmu}_1 =  (\mSigma_{11}^*)^{-1} \vmu_1^*
    - \mSigma_{11}^{-1}\mSigma_{12}  \mQ \mSigma_{21}\mSigma_{11}^{-1}\widetilde{\vmu}_1 
    $$
    
    \noindent After some cancellation we have $\widetilde{\vmu}_1 = \vmu_1^*$.
    
    
    \newpage 
    

    
    
    \item Consider the multivariate Lasso distribution
    $$
    p\left( \vx \right) \propto \exp\left(  -\tfrac{1}{2}\vx^T\mA\vx + \vb^T\vx - c\|\vx\|_1 \right)
    $$
    
    \noindent where $\mA\in\sS_d^+$ is a positive definite matrix of dimension $d$, $\vb\in\bR^2$ and $c>0$ with $\|\vx\|_1 = \sum_{j=1}^d |x_j|$. For the case $d=2$ 
    \begin{itemize}
        \item Find the normalizing constant.
        \item Find an expression for the expectation and the covariance.
    \end{itemize}
    


    
    
\end{enumerate}


\end{document}
